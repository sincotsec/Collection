\documentclass[fleqn, 12pt, a4paper, oneside, notitlepage]{book}
\usepackage{amsmath}
\usepackage{amssymb}
\usepackage[T2A]{fontenc}
\usepackage[russian]{babel}
\usepackage{indentfirst}

\pagestyle{plain}
\oddsidemargin=-60pt
\topmargin=-60pt
\textwidth=575pt
\textheight=760pt
\setcounter{MaxMatrixCols}{20}

\DeclareMathOperator{\sech}{sech}
\DeclareMathOperator{\csch}{csch}
\DeclareMathOperator{\arccot}{arccot}
\DeclareMathOperator{\arcsec}{arcsec}
\DeclareMathOperator{\arccsc}{arccsc}
\DeclareMathOperator{\arsinh}{arsinh}
\DeclareMathOperator{\arcosh}{arcosh}
\DeclareMathOperator{\artanh}{artanh}
\DeclareMathOperator{\arcoth}{arcoth}
\DeclareMathOperator{\arsech}{arsech}
\DeclareMathOperator{\arcsch}{arcsch}

\begin{document}
\title{Сборник формул}
\author{sincotsec}
\date{}
\maketitle

\makeatletter\renewcommand\chapter{%
\thispagestyle{empty}\global\@topnum\z@\@afterindentfalse
\secdef\@chapter\@schapter}\makeatother

\tableofcontents

\makeatletter\renewcommand\chapter{%
\if@openright\cleardoublepage\else\clearpage\fi
\thispagestyle{plain}\global\@topnum\z@\@afterindentfalse
\secdef\@chapter\@schapter}\makeatother

\chapter{Черновик}

\input Partitions/OmegaFunctions
\input Partitions/OmegaBinomial
\input Partitions/OnePlusX
\input Partitions/SumOfValues
\input Partitions/Polynom
\input Partitions/Coefficients
\input Partitions/NegativeDegreeOfPolynom
\input Partitions/DegreeSumOperator
\input Partitions/OperatorMultiplication
\input Partitions/OperatorVarieties
\input Partitions/SumAndMultiplicationOfFractions
\input Partitions/BinaryNumbersMultiplication
\input Partitions/Arrows
\input Partitions/Derivative
\input Partitions/Functions

\end{document}
