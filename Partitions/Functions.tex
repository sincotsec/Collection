% !TeX root = ../Collection.tex

\section{Функции}

\subsection{Логарифм}

\begin{equation*} \begin{aligned}
\ln{x} &= \lim_{a \to 0} \frac{x^a - 1}{a} \\
\end{aligned} \end{equation*}

\subsection{Показательная функция}

\begin{equation*}
\begin{aligned}
\exp{x} &= 
  1
+ x
+ \frac{x^2}{2!}
+ \frac{x^3}{3!}
+ \frac{x^4}{4!}
+ \ldots
\\
\exp{x} &=
  \omega_{0}(x)
+ \omega_{1}(x)
+ \omega_{2}(x)
+ \ldots
\\
\exp{x} &=
  \lim_{n \to \infty} \left(
  1 + \frac{x}{n}
  \right)^n
\\
\end{aligned}
\end{equation*}

\subsection{Тригонометрические и гиперболические} %112233 section ===

\begin{equation*}
\left|
\begin{aligned}
&\sin{x} & &\sinh{x} \\
&\cos{x} & &\cosh{x} \\
&\tan{x} & &\tanh{x} \\
&\cot{x} & &\coth{x} \\
&\sec{x} & &\sech{x} \\
&\csc{x} & &\csch{x} \\
\end{aligned}
\right.
\end{equation*}

\begin{equation*}
\left|
\begin{aligned}
\sinh{x} &= - i \sin{(i x)} & \sin{x} &= - i \sinh{(i x)}  \\
\cosh{x} &= \cos{(i x)} & \cos{x} &= \cosh{(i x)} \\
\tanh{x} &= - i \tan{(i x)} & \tan{x} &= - i \tanh{(i x)} \\
\coth{x} &= i \cot{(i x)} & \cot{x} &= i \coth{(i x)} \\
\sech{x} &= \sec{(i x)} & \sec{x} &= \sech{(i x)} \\
\csch{x} &= i \csc{(i x)} & \csc{x} &= i \csch{(i x)} \\
\end{aligned}
\right.
\end{equation*}

\begin{equation*}
\begin{aligned}
\cos{x} &= 
\left[1 - \left(\frac{2 x}{\pi} \right)^2 \right] 
\left[1 - \left(\frac{2 x}{3 \pi} \right)^2 \right]
\left[1 - \left(\frac{2 x}{5 \pi} \right)^2 \right] 
\left[1 - \left(\frac{2 x}{7 \pi} \right)^2 \right] 
\cdot \ldots \\
%
\sin{x} &= x 
\left[1 - \left(\frac{x}{\pi} \right)^2 \right] 
\left[1 - \left(\frac{x}{2 \pi} \right)^2 \right]
\left[1 - \left(\frac{x}{3 \pi} \right)^2 \right] 
\left[1 - \left(\frac{x}{4 \pi} \right)^2 \right] 
\cdot \ldots \\
%
\cosh{x} &= 
\left[1 + \left(\frac{2 x}{\pi} \right)^2 \right] 
\left[1 + \left(\frac{2 x}{3 \pi} \right)^2 \right]
\left[1 + \left(\frac{2 x}{5 \pi} \right)^2 \right] 
\left[1 + \left(\frac{2 x}{7 \pi} \right)^2 \right] 
\cdot \ldots \\
%
\sinh{x} &= x 
\left[1 + \left(\frac{x}{\pi} \right)^2 \right] 
\left[1 + \left(\frac{x}{2 \pi} \right)^2 \right]
\left[1 + \left(\frac{x}{3 \pi} \right)^2 \right] 
\left[1 + \left(\frac{x}{4 \pi} \right)^2 \right] 
\cdot \ldots \\
\end{aligned}
\end{equation*}

\begin{equation*}
\begin{aligned}
%---------------------------------------
\cos{x} 
&= 1 
- \frac{x^2}{2!} 
+ \frac{x^4}{4!} 
- \frac{x^6}{6!} 
+ \frac{x^8}{8!} 
- \frac{x^{10}}{10!} 
+ \ldots \\
%---------------------------------------
\sin{x} 
&= \frac{x}{1!} 
- \frac{x^3}{3!} 
+ \frac{x^5}{5!} 
- \frac{x^7}{7!} 
+ \frac{x^9}{9!} 
- \frac{x^{11}}{11!} 
+ \ldots \\
%---------------------------------------
\cosh{x} 
&= 1 
+ \frac{x^2}{2!} 
+ \frac{x^4}{4!} 
+ \frac{x^6}{6!} 
+ \frac{x^8}{8!} 
+ \frac{x^{10}}{10!} 
+ \ldots \\
%---------------------------------------
\sinh{x} 
&= \frac{x}{1!} 
+ \frac{x^3}{3!} 
+ \frac{x^5}{5!} 
+ \frac{x^7}{7!} 
+ \frac{x^9}{9!} 
+ \frac{x^{11}}{11!} 
+ \ldots \\
%---------------------------------------
\end{aligned}
\end{equation*}

\begin{equation*}
\begin{aligned}
%---------------------------------------
\sec{x} &=
\frac{1}{1 - \left(\displaystyle \frac{2 x}{\pi} \right)^2}
\frac{1}{1 - \left(\displaystyle \frac{2 x}{3 \pi} \right)^2}
\frac{1}{1 - \left(\displaystyle \frac{2 x}{5 \pi} \right)^2}
\frac{1}{1 - \left(\displaystyle \frac{2 x}{7 \pi} \right)^2}
\cdot \ldots \\
%---------------------------------------
\csc{x} &= \frac{1}{x} 
\frac{1}{1 - \left(\displaystyle \frac{x}{\pi} \right)^2}
\frac{1}{1 - \left(\displaystyle \frac{x}{2 \pi} \right)^2}
\frac{1}{1 - \left(\displaystyle \frac{x}{3 \pi} \right)^2}
\frac{1}{1 - \left(\displaystyle \frac{x}{4 \pi} \right)^2}
\cdot \ldots \\
%---------------------------------------
\sech{x} &=
\frac{1}{1 + \left(\displaystyle \frac{2 x}{\pi} \right)^2}
\frac{1}{1 + \left(\displaystyle \frac{2 x}{3 \pi} \right)^2}
\frac{1}{1 + \left(\displaystyle \frac{2 x}{5 \pi} \right)^2}
\frac{1}{1 + \left(\displaystyle \frac{2 x}{7 \pi} \right)^2}
\cdot \ldots \\
%---------------------------------------
\csch{x} &= \frac{1}{x} 
\frac{1}{1 + \left(\displaystyle \frac{x}{\pi} \right)^2}
\frac{1}{1 + \left(\displaystyle \frac{x}{2 \pi} \right)^2}
\frac{1}{1 + \left(\displaystyle \frac{x}{3 \pi} \right)^2}
\frac{1}{1 + \left(\displaystyle \frac{x}{4 \pi} \right)^2}
\cdot \ldots \\
%---------------------------------------
\end{aligned}
\end{equation*}

\begin{equation*}
\lim_{\mu \to 0}\frac{\tan{(x + \mu)} - \tan(x)}{\mu} = 1 + \tan^2(x)
\end{equation*}

\subsection{Обратные тригонометрические и обратные гиперболические}

\begin{equation*}
\left|
\begin{aligned}
&\arcsin{x} & &\arsinh{x} \\
&\arccos{x} & &\arcosh{x} \\
&\arctan{x} & &\artanh{x} \\
&\arccot{x} & &\arcoth{x} \\
&\arcsec{x} & &\arsech{x} \\
&\arccsc{x} & &\arcsch{x} \\
\end{aligned}
\right.
\end{equation*}

\subsection{Факториальные}

\subsubsection{Факториал}

\begin{equation*}
\begin{aligned}
\Pi(x) &= 1 \cdot 2 \cdot 3 \cdot 4 \cdot \ldots \cdot x
\\
\Pi(x + 1) &= \Pi(x) (x + 1)
\\
\Pi(x) &= \lim_{n \to \infty}
{
\cfrac{n^x}
{
\left(1 + \displaystyle \frac{x}{1} \right) 
\left(1 + \displaystyle \frac{x}{2} \right)
\left(1 + \displaystyle \frac{x}{3} \right) 
\cdot \ldots \cdot 
\left(1 + \displaystyle \frac{x}{n} \right)
}
} \\
\end{aligned}
\end{equation*}

\subsubsection{Отрицательная степень факториала}

\begin{equation*}
\begin{aligned}
\Phi(x) &= \frac{1}{1 \cdot 2 \cdot 3 \cdot 4 \cdot \ldots \cdot x} \\
\Phi(x + 1) &= \frac{\Phi(x)}{(x + 1)} \\
\Phi(x) &= \lim_{n \to \infty} \left[
n^{-x}
\left(1 + \displaystyle \frac{x}{1} \right) 
\left(1 + \displaystyle \frac{x}{2} \right)
\left(1 + \displaystyle \frac{x}{3} \right) 
\cdot \ldots \cdot 
\left(1 + \displaystyle \frac{x}{n} \right) \right] \\
\end{aligned}
\end{equation*}

\subsubsection{Логарифм отрицательной степени факториала}

\begin{equation*}
\begin{aligned}
\ln{\Phi(x)} &= - \ln{1} - \ln{2} - \ln{3} - \ln{4} + \ldots - \ln{x} \\
\ln{\Phi(x + 1)} &= \ln{\Phi(x)} - \ln{(x + 1)} \\
\ln{\Phi(x)} &= \lim_{n \to \infty}
{\left[
- x \ln{n}
+ \ln{\left(1 + \frac{x}{1} \right)} 
+ \ln{\left(1 + \frac{x}{2} \right)}
+ \ln{\left(1 + \frac{x}{3} \right)}
+ \ldots
+ \ln{\left(1 + \frac{x}{n} \right)}
\right]} \\
\end{aligned}
\end{equation*}

\subsection{Дзета-функция} %112233 subsection ===

\begin{equation*}
\begin{aligned}
\zeta(x) &= 1 + \frac{1}{2^x} + \frac{1}{3^x} + \frac{1}{4^x} + \ldots \\
\end{aligned}
\end{equation*}

\begin{equation*}
\begin{aligned}
\zeta(2) &= 
  \frac{1}{1^2}
+ \frac{1}{2^2} 
+ \frac{1}{3^2} 
+ \frac{1}{4^2}
+ \frac{1}{5^2} 
+ \ldots 
\\
%---------------------------------------
\zeta(3) &= 
  \frac{1}{1^3}
+ \frac{1}{2^3} 
+ \frac{1}{3^3} 
+ \frac{1}{4^3}
+ \frac{1}{5^3} 
+ \ldots 
\\
%----------------------------------------
\zeta(4) &= 
  \frac{1}{1^4}
+ \frac{1}{2^4} 
+ \frac{1}{3^4} 
+ \frac{1}{4^4}
+ \frac{1}{5^4} 
+ \ldots 
\\
%---------------------------------------
\end{aligned}
\end{equation*}
