% !TeX root = sincotsec.tex

\section{Функции}

\subsection{Логарифм}

\begin{equation*} \begin{aligned}
\ln{x} &= \lim_{a \to 0} \frac{x^a - 1}{a} \\
\end{aligned} \end{equation*}

\subsection{Показательная функция}

\begin{equation*}
\begin{aligned}
\exp{x} &= 
  1
+ x
+ \frac{x^2}{2!}
+ \frac{x^3}{3!}
+ \frac{x^4}{4!}
+ \ldots
= \lim_{n \to \infty} \left(
  1 + \frac{x}{n}
  \right)^n \\
\end{aligned}
\end{equation*}

\subsection{Обратные тригонометрические и обратные гиперболические}

\begin{equation*}
\left|
\begin{aligned}
&\arcsin{x} & &\arsinh{x} \\
&\arccos{x} & &\arcosh{x} \\
&\arctan{x} & &\artanh{x} \\
&\arccot{x} & &\arcoth{x} \\
&\arcsec{x} & &\arsech{x} \\
&\arccsc{x} & &\arcsch{x} \\
\end{aligned}
\right.
\end{equation*}

\subsection{Факториал}

\begin{equation*}
\begin{aligned}
\Pi(x) &= 1 \cdot 2 \cdot 3 \cdot 4 \cdot \ldots \cdot x
\\
\Pi(x + 1) &= \Pi(x) (x + 1)
\\
\Pi(x) &= \lim_{n \to \infty}
{
\cfrac{n^x}
{
\left(1 + \displaystyle \frac{x}{1} \right) 
\left(1 + \displaystyle \frac{x}{2} \right)
\left(1 + \displaystyle \frac{x}{3} \right) 
\cdot \ldots \cdot 
\left(1 + \displaystyle \frac{x}{n} \right)
}
} \\
\end{aligned}
\end{equation*}

\subsection{Отрицательная степень факториала}

\begin{equation*}
\begin{aligned}
\Phi(x) &= \frac{1}{1 \cdot 2 \cdot 3 \cdot 4 \cdot \ldots \cdot x} \\
\Phi(x + 1) &= \frac{\Phi(x)}{(x + 1)} \\
\Phi(x) &= \lim_{n \to \infty} \left[
n^{-x}
\left(1 + \displaystyle \frac{x}{1} \right) 
\left(1 + \displaystyle \frac{x}{2} \right)
\left(1 + \displaystyle \frac{x}{3} \right) 
\cdot \ldots \cdot 
\left(1 + \displaystyle \frac{x}{n} \right) \right] \\
\end{aligned}
\end{equation*}

\subsection{Логарифм отрицательной степени факториала}

\begin{equation*}
\begin{aligned}
\ln{\Phi(x)} &= - \ln{1} - \ln{2} - \ln{3} - \ln{4} + \ldots - \ln{x} \\
\ln{\Phi(x + 1)} &= \ln{\Phi(x)} - \ln{(x + 1)} \\
\ln{\Phi(x)} &= \lim_{n \to \infty}
{\left[
- x \ln{n}
+ \ln{\left(1 + \frac{x}{1} \right)} 
+ \ln{\left(1 + \frac{x}{2} \right)}
+ \ln{\left(1 + \frac{x}{3} \right)}
+ \ldots
+ \ln{\left(1 + \frac{x}{n} \right)}
\right]} \\
\end{aligned}
\end{equation*}

\subsection{Дзета-функция} %112233 subsection ===

\begin{equation*}
\begin{aligned}
\zeta(x) &= 1 + \frac{1}{2^x} + \frac{1}{3^x} + \frac{1}{4^x} + \ldots \\
\end{aligned}
\end{equation*}

\begin{equation*}
\begin{aligned}
\zeta(2) &= 
  \frac{1}{1^2}
+ \frac{1}{2^2} 
+ \frac{1}{3^2} 
+ \frac{1}{4^2}
+ \frac{1}{5^2} 
+ \ldots 
\\
%---------------------------------------
\zeta(3) &= 
  \frac{1}{1^3}
+ \frac{1}{2^3} 
+ \frac{1}{3^3} 
+ \frac{1}{4^3}
+ \frac{1}{5^3} 
+ \ldots 
\\
%----------------------------------------
\zeta(4) &= 
  \frac{1}{1^4}
+ \frac{1}{2^4} 
+ \frac{1}{3^4} 
+ \frac{1}{4^4}
+ \frac{1}{5^4} 
+ \ldots 
\\
%---------------------------------------
\end{aligned}
\end{equation*}